\documentclass[prb,twocolumn]{revtex4-1}
\begin{document}

\section{Introduction}
The study of topological phases of matter has been a major component of condensed matter research for the past several years. Among the many phases studied, the topological insulator (TI) is one of the most prominent. This is a three-dimensional phase of free fermions. Though it is insulating in the bulk, its topological behavior can be deduced from its unusual surface properties, in particular it has an odd number of Dirac cones on its surface. The topological insulator is an example of a symmetry-protected topological phase (SPT). Like all SPT's, it has short-ranged entanglement, which implies that is has a unique ground state on any closed manifold. This is in contrast to topologically ordered states like the fractional quantum Hall effect. The topological insulator has both charge conservation and time-reversal symmetry, and if either of these symmetries are broken it loses its topological properties.

One obvious extension of research into topological insulators is to consider the effects of interactions on their properties. This is however a difficult task. Many of the methods used to study TI's involve the properties of their band structure, and these methods obviously do not apply to the interacting case. As an introduction to this difficult problem, one can try to study an analog of the topological insulator, constructed of interacting bosons instead of fermions. 

The field of topological phases of bosons is relatively new, but much progress has been made. Chen, Liu, Gu and Wen\cite{WenScience,*WenPRB} have used group cohomology theory to determine which symmetries and dimensions can lead to non-trivial topological phases. However, this approach tells us little about the properties of these phases, which much be determined through other methods. One well-studied case is SPT's in two dimensions with $U(1)$ symmetry. A Chern-Simons field theory has been found to describe both the bulk and edges of these systems.\ref{LuVishwanath} These systems can be shown to have a Hall effect quantized to an even integer (in units of $e^2/h$) and gapless edge modes, and are therefore called `bosonic quantum Hall phases'. A powerful way to think about these phases is in the context of `flux-attachment', where we consider attaching a flux to each boson, in order to give the bosons non-trivial statistics.\cite{SenthilLevin} One drawback of the flux attachment technique is that it is not related to a microscopic model. To change this, in an earlier work we considered extending the symmetry to $U(1)\times U(1)$, i.e.~ two different species of bosons. Flux from one species of boson was bound to charge from the other species, and vice versa. In this way, we obtained a microscopic model which realizes the bosonic quantum Hall effect, and studied this model in Monte Carlo.\cite{FQHE} 

Interacting bosonic topological phases which are analogues of the topological insulator can also be considered. Senthil and Vishwanath\cite{SenthilVishwanath} have found effective field theories which can describe both the bulk and the surface of a three-dimensional topological phase with charge conservation and a $\mathbb{Z}_2$ symmetry. They find exotic behavior on the surface of their system which can be used to study the topological behavior in the bulk. The field theories can be constructed by adding an additional $SO(3)$ symmetry to the model, and binding monopoles of this symmetry to the bosons. Metliski and Fisher produced a construction which explicitly binds monopoles to bosons, and shown that it leads to the same topological phase. In that case the monopoles can be shown to come from the bosons themselves.\cite{Max} 

In both the two- and three-dimensional cases, the topological behavior can be thought of as coming from the binding of bosons to point topological defects. In two dimensions the topological defects were vortices of another species of bosons, this was the flux attachment. In three dimensions the topological defects come from monopoles.

In this work, we construct explicit models which realizes the interacting bosonic analog of the topological insulator. These model can be studied in Monte Carlo simulations. The main idea behind our constructions is the binding of monopoles to bosons, and condensation the resulting bound states. We present two different models which realize this physics. In the first model, described in Section \ref{section::Heisenberg}, we introduce additional $SO(3)$ degree of freedom in the form of Heisenberg spins. We can define `hedgehog' configurations from these spins, and these will serve as the topological defects which we bind to bosons. We introduce a term in our action which energetically binds hedgehogs to bosons, and we show that this term can lead to a phase (which we call the `binding phase') which is a condensate of these bound states.

Our next task is to show that the binding phase is indeed topological. In the two-dimensional case, one can demonstrate topological behavior by measuring a response which is topologically invariant: the Hall conductivity. In the three-dimensional case, no such response is known to exist. The most commonly cited evidence for topological behavior in the free fermion TI is the existence of an odd number of Dirac cones on the surface, but this diagnostic does not apply to our interacting case. Topological properties are often deduced by studying exotic physics on the surfaces of topological phases, and so we study the surface of the binding phase in Monte Carlo. We can determine the phase diagram of the surface and some of the properties of the surface phases, but find no smoking gun that would indicate topological behavior. 

The topological insulator, whether constructed from free fermions or interacting bosons, is beleived to have a topological term in its effective field theory which leads to a quantized magneto-electric effect. This effect can possibly be seen by looking at electromagnetic responses of the phase. One such response is the Witten effect, in which magnetic monopoles of an external gauge field become bound to half a quanta of electric charge. Another response is the Hall effect at the surface of the topological phase. This bulk topological field theory becomes a Chern-Simons theory at the surface of the system, which, when time-reversal is broken at the surface, should lead to a Hall effect quantized in units of one-half the value allowed in a purely two dimensional system. In the bosonic quantum Hall effect the Hall conductivity is quantized to be an even integer, so if the binding phase is topological it should have a Hall conductivity on its surface with odd-integer values.

In order to measure the above responses, we need to introduce external electromagnetic fields into our system. In the above model, this is complicated by our definition of hedgehog number. In the main text, we will see that hedgehog number is a discontinuous function of the Heisenberg spins. When the external fields are introduced, the action becomes a discontinuous function of these fields, which prevents us from using linear response theory to measure the above responses. To solve this problem, in Section \ref{section::CP1} we introduce a second model which binds monopoles to bosons. In this model, the additional $SO(3)$ degrees of freedom are described by CP$^1$ model instead of a Heisenberg model. The advantage of this model is that the monopole number can be defined as a continuous function, so this allows us to use linear response theory. We determine the phase diagram of both the bulk and the surface of this model. In the bulk, we introduce magnetic monopoles of the external electromagnetic field and measure one-half of a charge bound to them, precisely the predicted Witten effect. On the surface, we break the $\mathbb{Z}_2$ and find a Hall conductivity quantized to odd-integer values. 

In Monte Carlo studies of the `bosonic quantum Hall' phases, we found additional topological physics by binding multiple topological defects to multiple bosons. In Section \ref{section::multiple} we consider such a binding of multiple bosons or multiple monopoles. In the case of multiple bosons, we find that phases made up of bound states containing an even number of bosons are topologically trivial, while if the bound states contain an odd number of bosons the system is in the topological phase. When bound states contain multiple monopoles, we find that the system is in a topologically ordered phase.

\end{document}

